\documentclass[12pt,a4paper]{article}
\usepackage{CJKutf8}
\usepackage{amsmath,amssymb,amsfonts}
\usepackage{graphicx}
\usepackage{hyperref}
\usepackage{algorithm}
\usepackage{algorithmic}
\usepackage{listings}
\usepackage{xcolor}
\usepackage{geometry}

\geometry{a4paper,left=2.5cm,right=2.5cm,top=2.5cm,bottom=2.5cm}

\title{区块链共识算法作业(修复版V2)}
\author{学生姓名}
\date{\today}

\begin{document}
\begin{CJK}{UTF8}{gbsn}

\maketitle

\section{问题1:Algorand算法 (20分)}

考虑一个由$n = 3t + 1$个节点组成的委员会,其中$t$个节点是恶意的,剩余的$2t + 1$个节点是诚实的。每个节点以一个单比特开始。令$b_i$表示节点$i$的比特。诚实节点执行共识算法。恶意节点可能以任意方式偏离算法,包括被单个攻击者指导。节点在算法的每一轮中更新它们的$b_i$。我们希望共识算法对所有诚实节点终止,并满足:

\begin{itemize}
    \item 一致性:所有诚实节点的比特相同。
    \item 一致性:如果所有诚实节点以相同的比特开始,则它们以相同的比特结束。
\end{itemize}

考虑课堂上讨论的Algorand共识算法(第19讲),如图1所示。节点$i$记录它收到的0票数为$\#(0)$和1票数为$\#(1)$(包括自己的票)。

\begin{figure}[h]
    \centering
    \includegraphics[width=0.8\textwidth]{images/algorand_algorithm_improved.png}
    \caption{Algorand共识算法}
\end{figure}

使用$t = 3$(因此$n = 10$)模拟该过程。

\subsection{随机初始化}

让所有节点的比特随机初始化。恶意节点不应计票,而是简单地为随机比特投票。执行共识协议。输出每轮中所有10个节点的比特,直到所有诚实节点终止。标记每个节点终止的位置。简要解释输出。

\subsubsection{模拟结果}

我们使用Python实现了Algorand共识算法的模拟,严格遵循图1所示的多步骤过程。在随机初始化的情况下,模拟结果如下:

\begin{verbatim}
轮次	节点比特值			终止状态			步骤			投票[0,1]	公共硬币
0	0010001(0)(1)(1)	FFFFFFFFFF	1111111(1)(1)(1)	[0, 0]	None
1	0000000(0)(1)(0)	FFFFFFFFTT	2222222(1)(1)(1)	[6, 4]	1
2	0000000(1)(1)(0)	TTTTTTTTTT	2222222(1)(1)(1)	[7, 1]	0
所有诚实节点是否达成一致: True
\end{verbatim}

在上述输出中:
\begin{itemize}
    \item 第一列表示轮次,从0开始(初始状态)。
    \item 第二列表示每个节点的比特值,前7个是诚实节点,后3个是恶意节点(用括号标记)。
    \item 第三列表示每个节点是否已终止,"T"表示已终止,"F"表示未终止。
    \item 第四列表示每个节点当前所处的步骤(1, 2, 或 3)。
    \item 第五列表示该轮的投票结果,[0票数, 1票数]。
    \item 第六列表示该轮使用的公共硬币值(如果适用)。
\end{itemize}

\subsubsection{结果分析}

在初始状态(轮次0)中,7个诚实节点的比特值随机初始化为"0010001",3个恶意节点的比特值随机初始化为"011"。所有节点都处于第1步,未终止。

在第1轮中,所有节点进行投票。根据算法,诚实节点投票自己当前的比特值,而恶意节点随机投票。计票结果为[6, 4],表示有6票投给0,4票投给1。

由于$t=3$,终止条件为$2t+1 = 7$。由于$\#(0)=6 < 7$且$\#(1)=4 < 7$,没有达到第一步的终止条件。根据第一步的"otherwise"规则,所有诚实节点将比特值更新为0,并进入第2步。两个恶意节点随机终止。

在第2轮中,投票结果为[7, 1],表示有7票投给0,1票投给1。由于$\#(0)=7 = 2t+1$,达到了第二步的终止条件。所有诚实节点保持比特值为0并终止。此时所有节点都已终止。

最终,所有诚实节点成功达成一致,比特值均为0。这表明Algorand算法在存在恶意节点的情况下仍能有效地达成共识。

\subsection{诚实节点初始化为0}

将所有诚实节点的比特初始化为0。重复第1部分。

\subsubsection{模拟结果}

在所有诚实节点初始化为0的情况下,模拟结果如下:

\begin{verbatim}
轮次	节点比特值			终止状态			步骤			投票[0,1]	公共硬币
0	0000000(1)(0)(0)	FFFFFFFFFF	1111111(1)(1)(1)	[0, 0]	None
1	0000000(0)(0)(0)	TTTTTTTTFT	1111111(1)(2)(1)	[8, 2]	1
所有诚实节点是否达成一致: True
\end{verbatim}

\subsubsection{结果分析}

在初始状态(轮次0)中,所有7个诚实节点的比特值都初始化为0,3个恶意节点的比特值随机初始化为"100"。所有节点都处于第1步,未终止。

在第1轮中,所有节点进行投票。诚实节点投票自己当前的比特值(全为0),而恶意节点随机投票。计票结果为[8, 2],表示有8票投给0,2票投给1。

由于$t=3$,终止条件为$2t+1 = 7$。由于$\#(0)=8 > 7$,达到了第一步的终止条件。所有诚实节点保持比特值为0并终止。大部分恶意节点也终止,只有一个进入第2步。

最终,所有诚实节点成功达成一致,比特值均为0。这验证了Algorand算法的一致性属性:如果所有诚实节点以相同的比特开始,则它们以相同的比特结束。

\subsection{算法实现说明}

我们的实现严格遵循了图1所示的Algorand共识算法的多步骤过程:

\begin{enumerate}
    \item \textbf{第一步}:
        \begin{itemize}
            \item 节点计算收到的0票数 \#(0) 和1票数 \#(1)
            \item 如果 \#(0) $\geq 2t+1$,则节点设置 $b_i=0$ 并终止
            \item 如果 \#(1) $\geq 2t+1$,则节点设置 $b_i=1$ 并终止
            \item 否则(otherwise),节点设置 $b_i=0$ 并进入第二步
        \end{itemize}
    
    \item \textbf{第二步}:
        \begin{itemize}
            \item 节点再次计算收到的0票数 \#(0) 和1票数 \#(1)
            \item 如果 \#(0) $\geq 2t+1$,则节点设置 $b_i=0$ 并终止
            \item 如果 \#(1) $\geq 2t+1$,则节点设置 $b_i=1$ 并终止
            \item 否则(otherwise),节点设置 $b_i=1$ 并进入第三步
        \end{itemize}
    
    \item \textbf{第三步}:
        \begin{itemize}
            \item 节点使用公共硬币 $c$
            \item 节点设置 $b_i=c$ 并终止
        \end{itemize}
\end{enumerate}

这种多步骤的设计确保了算法在各种条件下的正确行为,而不仅仅是在特定的初始条件和投票模式下。

\section{问题2:最快违规 (50分)}

在这个问题中,我们研究最长链分叉选择规则的安全性。问题是:假设攻击者从时间0开始攻击系统,即在创世区块被挖掘后立即开始,创建第一个安全违规的最小预期时间或挖掘的区块数是多少?

\subsection{基本问题的清晰简洁表述}

基本问题可以表述如下:

在工作量证明的区块链系统中,考虑一个攻击者从创世区块开始进行私有挖掘攻击,目标是创造安全违规。系统中有诚实矿工(挖掘率为$h$)和恶意矿工(挖掘率为$a$),且满足$\frac{1}{a} > \frac{1}{h} + \Delta$,其中$\Delta$是区块传播延迟(在基本问题中$\Delta = 0$)。

攻击者可以采用带重置的私有挖掘策略:从创世区块开始挖掘私有链,并可以在任何时间点将基础区块重置为最高的诚实区块,然后在新基础上开始挖掘新的私有链。攻击目标是基础区块的第一个诚实子区块。

当目标区块在公共链中至少为$k$深,且包含当前基础的私有链是最长链时,安全违规发生。

问题是:设计一个重置策略,使得从时间0开始到第一次安全违规发生的预期时间最小。

\subsection{设计重置策略以保证违规}

为了保证违规,我们设计了几种不同的重置策略,并通过模拟比较它们的性能:

\begin{enumerate}
    \item \textbf{基本策略(不重置)}:从创世区块开始挖掘私有链,不进行重置。
    \item \textbf{基于落后程度的重置策略}:当私有链落后公共链超过一定阈值(如3个区块)时重置。
    \item \textbf{基于概率的重置策略}:每次检查时以一定概率(如0.1)重置。
    \item \textbf{基于时间的重置策略}:每隔一定时间(如50个时间单位)重置一次。
    \item \textbf{动态调整策略}:结合多种因素动态决定是否重置:
        \begin{itemize}
            \item 如果私有链落后太多(如5个区块),立即重置
            \item 如果目标区块接近$k$确认,不重置
            \item 否则,以较低概率(如0.05)重置
        \end{itemize}
\end{enumerate}

我们使用Python实现了这些策略的模拟,参数设置为$a = 0.3$,$h = 0.7$,$k = 5$,运行了100次模拟。模拟结果如下表所示:

\begin{figure}[h]
    \centering
    \includegraphics[width=0.8\textwidth]{images/reset_strategy_table.png}
    \caption{不同重置策略的比较}
\end{figure}

从结果可以看出,基于概率的重置策略在平均时间上表现最好(10.24),但成功率较低(0.06)。基于时间的重置策略成功率最高(0.24),但平均时间较长(13.36)。动态调整策略在成功率(0.21)和稳定性(标准差2.21)方面表现较好。

\subsection{设计最优重置策略}

设计最优重置策略的目标是使第一次违规的预期时间尽可能小。基于我们的模拟结果和理论分析,我们提出以下优化方向:

\subsubsection{理论分析}

从理论上讲,最优重置策略应该考虑以下因素:

\begin{enumerate}
    \item \textbf{公私链长度差距}:当私有链落后公共链太多时,继续在当前基础上挖掘成功概率很低,应该重置。
    \item \textbf{目标区块确认深度}:如果目标区块接近$k$确认,且私有链有赶上公共链的可能,应该坚持当前攻击而不重置。
    \item \textbf{挖掘率比例}:$a/h$的比值影响攻击成功的概率,当$a$接近$h$时,不重置的策略可能更有效。
    \item \textbf{已投入时间}:考虑已经投入的挖掘时间,避免过早放弃有希望的攻击。
\end{enumerate}

\subsubsection{马尔可夫决策过程方法}

一个更系统的方法是将问题建模为马尔可夫决策过程(MDP):

\begin{itemize}
    \item \textbf{状态}:$(l_h, l_a, d)$,其中$l_h$是公共链长度,$l_a$是私有链长度,$d$是目标区块在公共链中的深度。
    \item \textbf{动作}:继续当前攻击或重置基础区块。
    \item \textbf{转移概率}:基于$a$和$h$计算下一个区块由谁挖出的概率。
    \item \textbf{奖励}:当违规发生时为正值,其他情况为负值(表示时间成本)。
\end{itemize}

通过值迭代或策略迭代算法,可以求解最优策略。但这需要离散化连续时间模型,增加了计算复杂性。

\subsubsection{强化学习方法}

另一种方法是使用强化学习,特别是Q-learning或深度Q网络(DQN)来学习最优策略:

\begin{itemize}
    \item 智能体通过与环境交互学习最优策略
    \item 状态表示为公私链长度、目标深度等特征
    \item 奖励设计为违规发生时的正奖励减去时间成本
\end{itemize}

这种方法的优势是可以处理大状态空间,并且不需要精确的环境模型。

\subsubsection{实际挑战}

在实际实现最优重置策略时,我们面临以下挑战:

\begin{enumerate}
    \item \textbf{状态空间爆炸}:随着链长增加,状态空间呈指数增长。
    \item \textbf{模型精度}:实际区块链网络中的挖掘过程可能不完全符合指数分布假设。
    \item \textbf{参数敏感性}:最优策略对$a$、$h$、$k$等参数非常敏感。
    \item \textbf{计算复杂性}:求解精确的最优策略计算开销很大。
\end{enumerate}

\subsubsection{改进的动态策略}

基于上述分析,我们提出一个改进的动态重置策略:

\begin{enumerate}
    \item 当$l_h - l_a > \alpha \cdot \sqrt{l_h}$时重置($\alpha$是可调参数)
    \item 当目标区块深度$d > \beta \cdot k$且$l_a > \gamma \cdot l_h$时不重置($\beta$、$\gamma$是可调参数)
    \item 根据当前状态动态调整重置概率:$p_{reset} = \min(1, \max(0, \frac{l_h - l_a - \delta}{l_h}))$
\end{enumerate}

这个策略结合了确定性规则和随机性,可以适应不同的网络状态,有望在平均情况下获得更好的性能。

\section{总结}

在本作业中,我们研究了两个区块链共识相关的问题:

\begin{enumerate}
    \item 通过模拟验证了Algorand共识算法在存在恶意节点的情况下仍能有效达成共识,并满足一致性属性。我们严格遵循了作业中描述的多步骤过程,包括正确的终止条件和公共硬币机制。
    \item 分析了区块链中最快违规问题,设计并比较了多种重置策略,提出了理论上的最优策略框架。
\end{enumerate}

我们的研究表明,在区块链安全性分析中,理论分析和计算机模拟相结合的方法是非常有效的。未来工作可以进一步探索强化学习在优化攻击策略中的应用,以及考虑更复杂的网络模型(如非零传播延迟)对安全性的影响。

\begin{thebibliography}{9}
\bibitem{puterman1994}
M. L. Puterman, Markov Decision Processes: Discrete Stochastic Dynamic Programming. USA: John Wiley \& Sons, Inc., 1st ed., 1994.
\end{thebibliography}

\end{CJK}
\end{document}

